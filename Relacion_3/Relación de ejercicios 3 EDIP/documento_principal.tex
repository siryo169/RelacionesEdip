\documentclass[a4paper, 12pt]{article}

\usepackage{amsmath} %Todos los paquetes de matematicas
\usepackage{amsthm}
\usepackage{mathtools}
\providecommand{\abs}[1]{\lvert#1\rvert}
\providecommand{\norm}[1]{\lVert#1\rVert}
\usepackage{yhmath}
\usepackage[utf8]{inputenc}
\usepackage[spanish]{babel}
\usepackage{wrapfig} %Figuras flotantes
\usepackage{parselines}
\usepackage{enumitem}
\usepackage{xcolor}
\usepackage{graphicx}
\graphicspath{{images/}}
\usepackage{subcaption}
\usepackage[left=2cm,right=2cm,top=2cm,bottom=2cm]{geometry}
\usepackage{eurosym} %Euro, de nada misniños

\theoremstyle{definition}
\newtheorem{ej}{Ejercicio}



\title{\textbf{Relación de ejercicios 3 EDIP}}
\author{Carlos García, Bora Goker, Javier Gómez,  \\ Ana Graciani, J.Alberto Hoces}
\date{2020/2021}

\setlength{\parindent}{0px}

\begin{document}

\maketitle

\begin{ej}
Durante un año, las personas de una ciudad utilizan 3 tipos de transportes: metro (M), autobús (A), y coche particular (C). Las probabilidades de que durante el año hayan usado unos u otros transportes son:

\medskip

\begin{tabular}{c c c c c c c}
	M: 0.3; & A: 0.2; & C: 0.15; & M y A: 0.1; & M y C: 0.05; & A y C: 0.06; & M, A y C: 0.01
\end{tabular} 

\medskip

Calcular las probabilidades siguientes:
\begin{enumerate}[label=\textit{\alph*)}]
	\item que una persona viaje en metro y no en autobús.
\[
	P(M \cap \bar{A}) = P(M) - P(M \cap A) = 0.3 - 0.1 = \underline{0.2}
\]
	\item que una persona tome al menos dos medios de transporte.
\[
	P((M \cap A) \cup (M \cap C) \cup (A \cap C) \cup (M \cap A \cap C)) = P((M \cap A) \cup (M \cap C)) + P((A \cap C) \cup (M \cap A \cap C)) 
\]
\[
	- P(((M \cap A) \cup (M \cap C)) \cap ((A \cap C) \cup (M \cap A \cap C))) =
\]
\[
	P(M \cap A) + P(M \cap C) - P(M \cap A \cap C) + P(A \cap C) + P(M \cap A \cap C) - P(M \cap A \cap C) - P(M \cap A \cap C) =
\]
\[
	0.1 + 0.05 - 0.01 + 0.06 + 0.01 - 0.01 - 0.01 = \underline{0.19}
\]
	\item que una persona viaje en metro o en coche, pero no en autobús.
\[
	P((M \cup C) \cap \bar{A}) = P(M \cup C) - P((M \cup C) \cap A) = P(M) + P(C) - P(M \cap C) - [P((M \cap A) \cup (A \cap C))] =
\]
\[
	P(M) + P(C) - P(M \cap C) - P(M \cap A) - P(A \cap A) + P(M \cap A \cap C) =
\]
\[
	0.3 + 0.15 - 0.05 - 0.1 - 0.06 + 0.01 = \underline{0.25}
\]
	\item que viaje en metro, o bien en autobús y en coche.
\[
	P(M \cup (A \cap C)) = P(M) + P(A \cap C) - P(M \cap A \cap C) = 0.3 + 0.06 - 0.01 = \underline{0.35}
\]
	\item que una persona vaya a pie.
	Aquí se  nos plantean dos posibilidades: que o el ir a pie sea la única alternativa a los medios de transportes propuestos en el enunciado, o que no sea así. En el segundo caso el problema sería imposible de resolver puesto que faltarían datos. Resolvamoslo para el primer caso:
\[
	P(\overline{M \cup A \cup C}) = 1 - P(M \cup A \cup C) = 1 - [P(M \cup A) + P(C) - P((M \cup A) \cap C)] = 
\]
\[
	1 - [P(M) + P(A) - P(M \cap A) + P(C) - P((M \cap C) \cup (A \cap C))] =
\]
\[
	1 - [0.3 + 0.2 - 0.1 + 0.15 - 0.05 - 0.06 + 0.01] = \underline{0.55}
\]
\end{enumerate}
\end{ej}

\newpage

\begin{ej}
Sean \(A,B\) y \(C\) tres sucesos de un espacio probabilístico \(\Omega, \mathcal{A}, P)\), tales que \(P(A) = 0.4\), \(P(B) = 0.2\), \(P(C) = 0.3\), \(P(A \cap B) = 0.1\) y \((A \cup B) \cap C = \emptyset\). Calcular las probabilidades de los siguientes sucesos:

\medskip

\begin{enumerate}[label=\textit{\alph*)}]
	\item sólo ocurre \(A\).
\[
	P(A \cap \bar{B} \cap \bar{C}) = P(A \cap (\overline{B \cup C})) = P(A) - P(A \cap (B \cup C)) = P(A) - P((A \cap B) \cup (A \cap C)) =
\]
\[
	P(A) - P(A \cap B) - P(A \cap C) + P(A \cap B \cap C) = 0.4 - 0.1 - 0 - 0 = \underline{0.3}
\]
	\item ocurren los tres sucesos
\[
	P((A \cup B) \cap C) = P((A \cap C) \cup (B \cap C)) = 0 \Rightarrow P(A \cap C) = 0 \text{ y } P(B \cap C) = 0
\]
\[
	P(A \cap B \cap C) = \underline{0}
\]
	\item ocurren \(A\) y \(B)\) pero no \(C\).
\[
	P(A \cap B \cap \bar{C}) = P(A \cap B) - P(A \cap B \cap C) = P(A \cap B) = \underline{0.1}
\]
	\item por los menos dos ocurren.
\[
	P((A \cap B) \cup (B \cap C) \cup (A \cap C)) = P((A \cap B) \cup ((A \cup B) \cap C)) = P((A \cap B) \cup \emptyset) = P(A \cap B) = \underline{0.1}
\]
	\item ocurren dos y no más.
\[
	P((A \cap B) \cup (B \cap C) \cup (A \cap C)) = P(A \cap B) = \underline{0.1}
\]
	\item no ocurren más de dos.
\[
	P(\overline{A \cap B \cap C}) = 1 - P(A \cap B \cap C) = \underline{1}
\]
	\item ocurre por lo menos uno
\[
	P(A \cup B \cup C) = P(A \cup B) + P(C) - P((A \cup B) \cap C) = P(A) + P(B) + P(C) - P(A \cap B) = 
\]
\[
	0.4 + 0.2 + 0.4 - 0.1 = \underline{0.8}
\]
	\item ocurre sólo uno.
\[
	P((A \cap \bar{B} \cap \bar{C}) \cup (\bar{A} \cap B \cap \bar{C}) \cup (\bar{A} \cap \bar{B} \cap C)) = P((A \cap \bar{B} \cap \bar{C}) \cup (\bar{A} \cap B \cap \bar{C})) + P(\bar{A} \cap \bar{B} \cap C) - 0 = 
\]
\[
	P(A \cap \bar{B} \cap \bar{C}) + P(\bar{A} \cap B \cap \bar{C}) + P(\bar{A} \cap \bar{B} \cap C) - 0 - 0 = 
\]
\[
	P(A \cap \bar{B}) - P(A \cap \bar{B} \cap C) + P(\bar{A} \cap B) - P(\bar{A} \cap B \cap C) + P(\bar{A} \cap C) - P(\bar{A} \cap B \cap C) =
\]
\[
	P(A) - P(A \cap B) - P(A \cap C) + P(A \cap B \cap C) + P(B) - P(A \cap B) - P(B \cap C) +
\]
\[
	+ P(A \cap B \cap C) + P(C) - P(A \cap C) - P(B \cap C) + P(A \cap B \cap C) =
\]
\[
	P(A) - P(A \cap B) + P(B) - P(A \cap B) + P(C) = 0.4 - 0.1 + 0.2 - 0.1 + 0.3 = \underline{0.7}
\]
	\item no ocurre ninguno.
\[
	P(\overline{A \cup B \cup C}) = 1 - P(A \cup B \cup C) = 1 - 0.8 = \underline{0.2}
\]
\end{enumerate}
\end{ej}

\newpage

\begin{ej}
Se sacan dos bolas sucesivamente sin devoluci ́on de una urna que contiene 3 bolas rojas distinguibles y 2 blancas distinguibles. \\
\begin{enumerate}
    \item[a) ]Describir el espacio de probabilidad asociado a este experimento.\\
    $\Omega =
    \{b_i b_j; i,j=1,2 ;i \neq j \} \cup \{ b_i r_j; i=1,2 ; j= 1,2,3\} \cup  \{ r_i b_j; i=1,2 ; j= 1,2,3\} \cup \\ \cup \{ r_i r_j; i,j=1,2,3; i\neq j\} $ 
    $$\#(\Omega) = 20$$
    \item[b) ] Descomponer en sucesos elementales los sucesos:la primera bola es roja,la segunda bola esblanca y calcular la probabilidad de cada uno de ellos. \\
    $P(1^a \ roja) = P(1^a \ roja \ 2^a \ roja) + P(1^a \ roja \ 2^a \ blanca) = P(r_i r_j; i,j=1,2,3; i\neq j) +$\\$+ \ P(r_i b_j; i=1,2,3 ; j= 1,2) = \dfrac{2*3}{20} + \dfrac{2*3}{20} = \dfrac{3}{5}$ \\
    $P(2^a \ blanca) = P(1^a \ roja \ 2^a \ blanca) + P(1^a \ blanca \ 2^a \ blanca) = P(r_i b_j; i=1,2,3 ; j= 1,2) +$\\$+ \ P(b_i b_j; i,j=1,2 ;i \neq j) = \dfrac{2*1}{20} + \dfrac{3*2}{20} = \dfrac{2}{5}$
    \item[c) ] ¿Cúal es la probabilidad de que ocurra alguno de los sucesos considerados en el apartado anterior?\\
    Usando el principio de inclusión-exclusión: \\
    $P(1^a \ roja \cup 2^a \ blanca) = P(1^a \ roja) + P(2^a \ blanca) - P(1^a \ roja 2^a \ blanca) =$\\$= 1- \dfrac{3*2}{20} = 0.7$
    
\end{enumerate}
\end{ej}

\begin{ej}
Una urna contiene a bolas blancas y b bolas negras. ¿Cuál es la probabilidad de que al extraer
dos bolas simultáneamente sean de distinto color?

\medskip

Aunque en el enunciado del problema se nos especifique que ambas bolas se extraen de la urna simultáneamente, podemos considerar la extracción de una bola y luego de otra sin reemplazamiento con el fin  de facilitar el cálculo de la probabilidad que se nos pide. Nombraremos B y N a los sucesos ``sacar bola blanca'' y ``sacar bola negra'' respectivamente. Que sean de distinto color significa que primero se saca una bola negra y luego una blanca o bien una blanca y después una negra. Por lo tanto, el cálculo de la probabilidad del suceso pedido será igual a la suma de estas dos probabilidades:

\[
	P(B/N)\cdot P(N) \Rightarrow \text{Que salga blanca después de sacar una negra}
\]

\[
	P(N/B)\cdot P(B) \Rightarrow \text{Que salga negra después de sacar una blanca}
\]

\[
	P(\text{2 bolas de distinto color}) = P(B/N)\cdot P(N) + P(N/B)\cdot P(B) =
\]

\[
	= \frac{b}{n+b-1}\cdot\frac{n}{n+b} + \frac{n}{n+b-1}\cdot\frac{b}{n+b} = \frac{2nb}{n^2+b^2+2nb-n-b}
\]

\end{ej}

\medskip

\begin{ej}
Una urna contiene 5 bolas blancas y 3 rojas. Se extraen 2 bolas simultáneamente. Calcular la
probabilidad de obtener:
\begin{enumerate}[label=\textit{\alph*)}]
\item dos bolas rojas
\item dos bolas blancas
\item una blanca y otra roja.
\end{enumerate}

No influye el orden en el que sacamos las bolas, los pares se sacan simultáneamente, y no se pueden repetir elementos porque no podemos sacar la misma bola 2 veces, luego para calcular las situaciones posibles calculamos las combinaciones sin repetición. \\
Tenemos 8 bolas en total y sacamos 2; $\qquad n=8 , \quad p=2$
\[C^p_n = C^2_8 = \binom{8}{2} = 28\]
\begin{enumerate}[label=\textit{\alph*)}]
\item No influye el orden y no se pueden repetir elementos $\longrightarrow$ combinaciones sin repetición

Hay 3 bolas rojas $\quad\Longrightarrow\quad$ situaciones favorables: $\quad C_3^2 = \binom{3}{2} = 3$
\[P(\text{dos bolas rojas}) = \frac{C_3^2}{C_8^2} = \frac{3}{28} = 0.107\]
\item \[P(\text{dos bolas blancas}) = \frac{C_5^2}{C_8^2} = \frac{\binom{5}{2}}{\binom{8}{2}} = \frac{10}{28} = \frac{5}{14} = 0.357\]
\item 5 bolas blancas $\quad\Longrightarrow\quad \text{situaciones favorables 1 bola blanca:}\quad C_5^1 = \binom{5}{1} = 5$ \\
3 bolas rojas $\quad\Longrightarrow\quad \text{situaciones favorables 1 bola roja:}\quad C_3^1 = \binom{3}{1} = 3$
\[P(\text{una bola blanca y otra roja}) = \frac{C_5^2 \cdot C_3^2}{C_8^2} = \frac{5\cdot3}{28} = \frac{15}{28} = 0.536\]
\end{enumerate}

\end{ej}

\medskip

\begin{ej}
\end{ej}

\begin{ej}
Se consideran los 100 primeros números naturales. Se sacan 3 al azar.\\
\begin{enumerate}
    \item[a) ] Calcular la probabilidad de que en los 3 primeros números obtenidos no exista ningún cuadrado perfecto. \\
    Si se hacen sin repetición: \\ $P(\text{No cuadrado perfecto}) = \dfrac{{90 \choose 3}}{{100 \choose 3}}  = \frac{90\cdot89\cdot88}{3\cdot2}= \dfrac{117480}{161700} = 0.727$  \\
    Si se hacen con repetición: \\  $P(\text{No cuadrado perfecto}) = \dfrac{90^3}{100^3}=\dfrac{72900}{1000000} = 0.729$
    \item[b) ] Calcular la probabilidad de que exista al menos un cuadrado perfecto.
    Si se hacen sin repetición: \\ $P(\text{Cuadrado perfecto}) = 1 - P(\text{No cuadrado perfecto}) = 1 - 0.727 = 0.273$ \\
    Si se hacen con repetición: \\ $P(\text{Cuadrado perfecto}) = 1 - P(\text{No cuadrado perfecto}) = 1 - 0.729 = 0.271$
    \item[c) ] Calcular la probabilidad de que exista un solo cuadrado perfecto, de que existan dos y de que tres lo sean.\\
    Si se hacen sin repetición: \\ 
    $P(\text{Un solo cuadrado perfecto}) = \dfrac{{90 \choose 2} \cdot {10 \choose 1}}{{100 \choose 3}} = \dfrac{40050}{161700} = 0.248$ \\
    $P(\text{Dos cuadrados perfectos}) = \dfrac{{90 \choose 1} \cdot {10 \choose 2}}{{100 \choose 3}} = \dfrac{4050}{161700} = 0.025$\\
    $P(\text{Tres cuadrados perfectos}) = \dfrac{{10 \choose 3}}{{100 \choose 3}} = \dfrac{120}{161700} = 0.001$\\
    Si se hacen con repetición:\\
    $P(\text{Un solo cuadrado perfecto}) = \dfrac{90^2\cdot 10}{100^3} = \dfrac{81000}{1000000} = 0.081$ \\
    $P(\text{Dos cuadrados perfectos}) = \dfrac{90 \cdot 10^2}{100^3} = \dfrac{9000}{1000000} = 0.009$\\
    $P(\text{Tres cuadrados perfectos}) = \dfrac{10^3}{100^3} = \dfrac{1000}{1000000} = 0.001$\\
    
\end{enumerate}
\end{ej}

\medskip

\begin{ej}
En una carrera de relevos cada equipo se compone de 4 atletas. La sociedad deportiva de un colegio cuenta con 10 corredores y su entrenador debe formar un equipo de relevos que disputará el
campeonato, y el orden en que participarán los seleccionados.
\begin{enumerate}[label=\textit{\alph*)}]
\item ¿Entre cuántos equipos distintos habrá de elegir el entrenador si los 10 corredores son de
igual valía? (Dos equipos con los mismos atletas en orden distinto se consideran diferentes)

\medskip

Influye el orden de los atletas en el equipo, no intervienen los 10 corredores del colegio en el equipo y un mismo corredor no puede ocupar dos puestos en un equipo, por lo que no se pueden repetir $\longrightarrow$ variaciones sin repetición \\
Calculamos el número de variaciones sin repetición, que será el número de equipos distintos posibles; $n = 10,\qquad p = 4$
\[V^4_{10} = \frac{10!}{(10-4)!} = 7\cdot8\cdot9\cdot10 = 5040\text{  equipos}\]
\item Calcular la probabilidad de que un alumno cualquiera sea seleccionado.

\medskip

$A$: ser seleccionado un alumno cualquiera \\
situaciones favorables: $V_4^1 \cdot V_9^3 \quad; \qquad\qquad$ situaciones posibles: $V_{10}^4$
\[P(A) = \frac{V_4^1 \cdot V_9^3}{V_{10}^4} = \frac{\frac{4!}{(4-1)!}\frac{9!}{(9-3)!}}{\frac{10!}{(10-4)!}} = \frac{\frac{4!}{3!}\frac{9!}{6!}}{\frac{10!}{6!}} = \frac{4\cdot 9!}{10!} = \frac{4}{10} = 0.4\]

\end{enumerate}
\end{ej}

\medskip

\begin{ej}
Una tienda compra bombillas en lotes de 300 unidades. Cuando un lote llega, se comprueban 60
unidades elegidas al azar, rechazándose el envío si se supera la cifra de 5 defectuosas. ¿Cuál es
la probabilidad de aceptar un lote en el que haya 10 defectuosas?

\medskip

Estamos ante un problema de combinatoria. Haremos uso del esquema dado en clase:

\medskip

¿Intervienen todos los elementos? $\longrightarrow$ No, pues de cada 300 uds. se toman 60.
\newline
¿Influye el orden? $\longrightarrow$ No, ya que lo único que nos importa es si las unidades son o no defectuosas, las mismas 60 unidades dispuestas en distinto orden no va a alterar la decisión de descartar el lote.
\newline
¿Se pueden repetir los elementos? $\longrightarrow$ No, ya que no se puede escoger la misma bombilla dos veces.

Por lo tanto, estamos ante combinaciones sin repetición. El número total de combinaciones posibles será $C_{300}^{60} = \frac{300!}{60!240!}$. Ahora necesitamos hallar las probabilidades de los sucesos favorables a que el lote sea aceptado, es decir, que de las 60 uds. escogidas haya 0, 1, 2, 3, 4 o 5 bombillas defectuosas. Llamemos A, B, C, D, E y F a los sucesos de sacar 0, 1, 2, 3, 4 y 5 piezas defectuosas entre las 60 escogidas respectivamente. Si tenemos 10 bombillas defectuosas entre las 300, habrá 290 bombillas perfectas. Para calcular la probabilidad de que en las 60 escogidas haya 0 defectuosas (suceso A) tendremos que hallar el cociente entre $C_{290}^{60}$ (combinaciones de 290 elementos tomados de 60 en 60) y $C_{300}^{60}$ (esto se debe a la regla de Laplace):

\[
    P(A) = \frac{C_{10}^{0} \cdot C_{290}^{60}}{C_{300}^{60}} = \frac{\binom{10}{0} \binom{290}{60}}{\binom{300}{60}} = \underline{0.10333} 
\]

Para la probabilidad de que en las 60 haya una defectuosa (suceso B) hemos de tener en cuenta que habrá 1 bombilla defectuosa y 59 buenas, por lo que por cada combinación de bombillas defectuosas que se han fijado en las 60 habrá $C_{290}^{59}$ combinaciones posibles de bombillas buenas, y concluimos que la probabilidad del suceso B es:

\[
    P(B) = \frac{C_{10}^{1} \cdot C_{290}^{59}}{C_{300}^{60}} = \frac{\binom{10}{1} \binom{290}{59}}{\binom{300}{60}} = \underline{0.26838}
\]

Repetimos los mismos razonamientos con los sucesos C, D, E y F:

\[
    P(C) = \frac{C_{10}^{2} \cdot C_{290}^{58}}{C_{300}^{60}} = \frac{\binom{10}{2} \binom{290}{58}}{\binom{300}{60}} = \underline{0.307137} \hspace{0.3cm}
    P(D) = \frac{C_{10}^{3} \cdot C_{290}^{57}}{C_{300}^{60}} = \frac{\binom{10}{3} \binom{290}{57}}{\binom{300}{60}} = \underline{0.20388}
\]

\[
    P(E) = \frac{C_{10}^{4} \cdot C_{290}^{56}}{C_{300}^{60}} = \frac{\binom{10}{4} \binom{290}{56}}{\binom{300}{60}} = \underline{0.08691} \hspace{0.3cm}
    P(F) = \frac{C_{10}^{5} \cdot C_{290}^{55}}{C_{300}^{60}} = \frac{\binom{10}{5} \binom{290}{55}}{\binom{300}{60}} = \underline{0.024853}
\]

Y calculamos la probabilidad de aceptar el lote como la suma de todas las probabilidades calculadas:

\[
    P(\text{Aceptar lote con 10 piezas defectuosas}) = \underline{0.99449}
\]


\end{ej}

\begin{ej}
\end{ej}

\end{document}
