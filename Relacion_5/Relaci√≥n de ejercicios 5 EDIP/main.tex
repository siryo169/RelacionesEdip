\documentclass[a4paper, 12pt]{article}

\usepackage{amsmath} %Todos los paquetes de matematicas
\usepackage{amsthm}
\usepackage{amsfonts}
\usepackage{mathtools}
\providecommand{\abs}[1]{\lvert#1\rvert}
\providecommand{\norm}[1]{\lVert#1\rVert}
\usepackage{yhmath}
\usepackage[utf8]{inputenc}
\usepackage[spanish]{babel}
\usepackage{wrapfig} %Figuras flotantes
\usepackage{parselines}
\usepackage{enumitem}
\usepackage{xcolor}
\usepackage{graphicx}
\usepackage{tikz}
\usetikzlibrary{positioning}
\usepackage{pgfplots}
\graphicspath{{images/}}
\usepackage{subcaption}
\usepackage[left=2cm,right=2cm,top=2cm,bottom=2cm]{geometry}
\usepackage{eurosym} %Euro, de nada misniños

\theoremstyle{definition}
\newtheorem{ej}{Ejercicio}



\title{\textbf{Relación de ejercicios 5 EDIP}}
\author{Carlos García, Bora Goker, Javier Gómez,  \\ Ana Graciani, J.Alberto Hoces}
\date{2020/2021}

\setlength{\parindent}{0px}

\begin{document}

\maketitle

\begin{ej}
Sea $X$ una variable aleatoria con función masa de probabilidad $P(X=i)= ki; \hspace{0.1cm} i = 1, . . . , 20$.

\begin{enumerate}
    \item[a)] Determinar el valor de $k$, la función de distribución y las siguientes probabilidades:
    
    \[
    P (X = 4), \hspace{0.1cm} P (X < 4), \hspace{0.1cm} P (3 \leq X \leq 10), \hspace{0.1cm} P (3 < X \leq 10), \hspace{0.1cm} P (3 < X < 10)
    \]
    
    Para poder determinar el valor de $k$, hemos de tener en cuenta que estamos ante una variable aleatoria discreta $X$, y la función masa de probabilidad de una variable aleatoria debe cumplir que $\sum_{i} P(X=i)=1$. En nuestro caso concretamente es $\sum_{i=1}^{20} P(X=i)=1$, lo cual se traduce en la siguiente ecuación: 
    
    \[
        \sum_{i=1}^{20} P(X=i)=1 \rightarrow 210k = 1 \rightarrow k = \frac{1}{210}
    \]
    
    Para hallar la función de distribución tendremos en cuenta su definición: 
    \[
    F_{X}(i) = P(X \leq i) = \sum_{j=1}^{i} P(X=j) \hspace{0.3cm} \forall i \in \{1,...,20\}
    \]
    
    Por lo que la función de distribución de la variable aleatoria X estará definida como:
    
    \[
    F_{X}(i)= \left\{ \begin{array}{lcc}
             0 &   si  & i \leq 1 \\
             \\ \sum_{j=1}^{i} P(X=j) = \sum_{j=1}^{i} \frac{1}{210} \cdot j &  si & 1 \leq i \leq 20 \\
             \\ 1 &  si  & x > 20
             \end{array}
   \right.
    \]
    
    Y ahora podemos emplear esta función para hallar las probabilidades pedidas (cuando sea la probabilidad de que la variable aleatoria tome un valor concreto, usaremos la función masa de probabilidad):
    
    \[
    P(X=4) = \frac{1}{210} \cdot 4 = \frac{2}{105} = 0.0190476
    \]
    
    \[
    P(X<4) = P(X\leq 3) = F_{X}(3) = \frac{1}{210} + \frac{2}{210} + \frac{3}{210} = \frac{1}{35} = 0.0285714
    \]
    
    \[
    P(3\leq X \leq 10) = P(X\leq10) - P(X<3) = P(X\leq10) - P(X\leq2) = F_{X}(10)-F_{X}(2)=
    \]
    
    \[
    = \sum_{j=3}^{10} \frac{1}{210} \cdot j = \frac{26}{105} = 0.247619
    \]
    
    \[
    P(3 < X \leq 10) = P(X\leq10) - P(X \leq 3) = F_{X}(10)-F_{X}(3)= \sum_{j=4}^{10} \frac{1}{210} \cdot j = \frac{7}{30} = 0.233333
    \]
    
     \[
    P(3 < X < 10) = P(X<10) - P(X\leq3) = P(X\leq9) - P(X\leq3) = F_{X}(9)-F_{X}(3)=
    \]
    
    \[
    = \sum_{j=4}^{9} \frac{1}{210} \cdot j = \frac{13}{70} = 0.1857142
    \]
    
    \item[b)] Supongamos que un jugador gana 20 monedas si al observar esta variable obtiene un valor
    menor que 4, gana 24 monedas si obtiene el valor 4 y, en caso contrario, pierde una moneda.
    Calcular la ganancia esperada del jugador y decir si el juego le es favorable.
    
    Definimos una nueva variable aleatoria $Y$ que será el número de monedas obtenidas en función de los valores que toma la variable aleatoria $X$:
    
    \[
    Y= h(X) = \left\{ \begin{array}{lcc}
             20 &   si  & 1 < X < 4 \\
             \\ 24 &  si & X = 4 \\
             \\ -1 &  si  & 4 < X \leq 20
             \end{array}
   \right.
    \]
    
    Y calculamos la probabilidad de que la variable $Y$ tome cada uno de sus valores posibles para que podamos hallar su esperanza:
    \[
    P(Y=20)=P(X<4)=P(X\leq3)=F_{X}(3) = \frac{1}{210} + \frac{2}{210} + \frac{3}{210} = \frac{1}{35} = 0.0285714
    \]
    
    \[
    P(Y=24)=P(X=4)= \frac{1}{210} \cdot 4 = \frac{2}{105} = 0.0190476
    \]
    
    \[
    P(Y=-1)=P(X>4)=1-P(X \leq 4) = 1-P(X<4)-P(X=4)= \frac{20}{21} = 0.95238
    \]
    
    Ya podemos hallar la esperanza matemática de $Y (y_{1} = 20, y_{2} = 24, y_{3} = -1)$ :
    
    \[
    E[Y]=\sum_{i}^3 y_{i}P(Y=y_{i})=\frac{20}{35}+\frac{48}{105}-\frac{20}{21} = \frac{8}{105} = 0.0761904
    \]
    
    Como la esperanza de $Y$ es mayor que 0, el juego es favorable al jugador.
\end{enumerate}
\end{ej}

\newpage

\begin{ej}

\end{ej}

\begin{ej}
El número de lanzamientos de una moneda hasta salir cara es una variable aleatoria con distribución \(P(X=x) = 2^{-x}\); \(x=1,2,\dotsc\).

\begin{enumerate}[label=\textit{\alph*)}]
	\item Probar que la función masa de probabilidad está bien definida.
	
Para que la función masa esté bien defindida, se tiene que cumplir que:
\[
	P(X=x_i) \geq 0 \quad \forall i \in \mathbb{N} \qquad \sum_{i=1}^{\infty} P(X=x_i) = \sum_{i=1}^{\infty} 2^{-x} \overset{\text{Serie Geométrica}}{\Rightarrow} \frac{\frac{1}{2}}{1-\frac{1}{2}} = 1
\]

La primera condición se cumple siempre, puesto que que la función tiene la expresión \(2^{-x}\), la cual es positiva \(\forall x \in \mathbb{R}\).

La segunda condición se demuestra observando que es una serie geométrica, las cuales convergen si y solo si su razón \(|r| < 1\). En este caso:
\[
	|r| = \frac{1}{2} < 1 \Rightarrow \text{La serie converge}
\]

Se cumplen las dos condiciones, así, afirmamos que la función masa de probabilidad está bien definida.

	\item Calcular la probabilidad de que el número de lanzamientos necesarios para salir cara esté entre 4 y 10.
	
Esto es calcular
\[
	P(4 \leq x \leq 10) = \sum_{i=4}^{10}p_i = 2^{-4} \cdot 2^{-5} \cdot 2^{-6} \cdot 2^{-7} \cdot 2^{-8} \cdot 2^{-9} \cdot 2^{-10} = \frac{127}{1024}
\]

	\item Calcular los cuartiles y la moda de la distribución, interpretando los valores.
	
Como se ha visto en otros temas, sabemos que el percentil de orden \(n\) se calcula:
\[
	P(X \leq x_i) = \frac{n}{100} \Rightarrow P_n = x_i.
\]

Así, 
\[
	P(X \leq x_i) = 0.25 \Rightarrow x_i = 1 \Rightarrow Q_1 = 1 \qquad P(X \leq x_i) = 0.5 \Rightarrow x_i = 1 \Rightarrow Q_2 = 1
\]
\[
	P(X \leq x_i) = 0.75 \Rightarrow P(X=1) + P(X=2) \geq \frac{3}{4} \Rightarrow Q_3 = 2
\]

En una variable aleatoria discreta, definimos la moda como:
\[
	Mo = x_i : P(X=x_i) = \text{máx} \{p_i\}
\]
Al ser la función masa de probabilidad decreciente, el máximo valor se encuentra en 1. Así:
\[
	x_1 = 1 \Rightarrow Mo = 1
\]

	\item Calcular función generatriz de momentos y, a partir de ella, el número medio de lanzamientos necesarios para salir cara y la desviación típica.
	
Definimos la función generatriz de momentos como
\[
M_x(t) = E[e^{tX}] \quad \forall t \in (t_0, t_1) \text{ centrado en el origen}
\]

\[
	M_x(t) = \sum_{x=1}^{\infty} e^{tx} \frac{1}{2^x} = \sum_{x=1}^{\infty} \left( \frac{e^t}{2} \right)^x = \frac{\frac{e^t}{2}}{1- \frac{e^t}{2}}
\]
Observamos que esta serie es absolutamente convergente \(\forall t < \log (2)\), por lo que converge en un entorno de cero, y por tanto, existe la función generatriz de momentos.

Ahora calculemos su esperanza:
\[
	E[X] = M'_X(t) = \frac{\frac{e^t}{2}(1-\frac{e^t}{2}) - \frac{e^t}{2} \cdot - \frac{e^t}{2}}{(1-\frac{e^t}{2})^2}
\]
\[
	M'_X(0) = 2 = E[X]
\]

Por último calculamos su desviación típica:
\[
	E[X^2] = M''_X(t) = \frac{2e^t (e^t -2)^2 - 2e^t (e^t-2)e^t}{(e^t-2)^4}
\]
\[
	M''_X(0) = 6 \Rightarrow Var|X| = E[X^2] - E[X]^2 = 2 \Rightarrow \sigma_x = \sqrt{2}
\]

\end{enumerate}

\end{ej}

\begin{ej}

\end{ej}

\begin{ej}

\end{ej}

\begin{ej}
Sea \(X\) una variable aleatoria con función de densidad
\[
	f(x) = \left\{ \begin{array}{cc}
	\frac{2x-1}{10} & 1 < x \leq 2 \\
	& \\
	0.4 & 4 < x \leq 6
	\end{array} \right.
\]

\begin{enumerate}[label=\textit{\alph*)}]
	\item Calcular \(P(1.5 < X \leq 2)\), \(P(2.5 < X \leq 3.5)\), \(P(4.5 \leq X \leq 5.5)\), \(P(1.2 < X \leq 5.2)\).
	
Primero calculamos:

\[
	\int_{a}^{t} \frac{2x-1}{10}dx = \left. \frac{x^2 - x}{10} \right|_{a}^{t} \quad t \in [1,2]
\]
\[
	\int_{a}^{t} 0.4x = \left. 0.4x \right|_{a}^{t} \quad t \in [4,6]
\]

Así:
\[
	P(1.5 < X \leq 2) = \left. \frac{x^2 - x}{10} \right|_{1.5}^{2} = \frac{1}{8} \qquad P(2.5 < X \leq 3.5) = 0
\]
\[
P(4.5 \leq X \leq 5.5) = \left. 0.4x \right|_{4.5}^{5.5} = \frac{2}{5} \qquad P(1.2 < X \leq 5.2) = \left. \frac{x^2 - x}{10} \right|_{1.2}^{2} + \left. 0.4x \right|_{4}^{5.2} = \frac{82}{125}
\]

	\item Dar la expresión general de los momentos no centrados y deducir el valor medio de \(X\).
	
Primero daremos la expresión del momento de orden \(k\) no centrado:
\[
	m_k = \int_{-\infty}^{+\infty} x^k f(x) dx = \int_{-\infty}^{1} 0 dx + \int_{1}^{2} x^k \frac{2x-1}{10}dx + \int_{2}^{4} 0 dx + \int_{4}^{6} 0.4x^kdx + \int_{6}^{+\infty} 0 dx =
\]
\[
	= \int_{1}^{2} \frac{2x -1 }{10} dx + \int_{4}^{6} 0.4x^k dx
\]

Y por tanto podemos afirmar que la esperanza (valor medio) es:
\[
	E[X] = m_1 = \int_{1}^{2} x \frac{2x-1}{10} dx + \int_{4}^{6} 0.4 x dx = \left. \frac{2x^3}{30} - \frac{x^2}{20} \right|_{1}^{2} + \left. \frac{x^2}{5} \right|_{4}^{6} = \frac{259}{60} 
\]

	\item Calcular la función generatriz de momentos de \(X\).
	
Definimos la función generatriz de momentos como
\[
	M_X(t) = E[e^{tX}] \quad \forall t \in (t_0, t_1) \text{ que contenga al origen}
\]

Así
\[
	M_X(t) = \int_{-\infty}^{+\infty} e^{tx} f(x) dx = \int_{1}^{2} e^{tx} \frac{2x-1}{10} dx + \int_{4}^{6} 0.4e^{tx} dx = 
\]
\[
	\frac{1}{10} \cdot \left. \left( 2 \left( x e^{tx} - \frac{e^{tx}}{t^2} \right) - \frac{e^{tx}}{t} \right) \right|_{1}^{2}  + 0.4 \left. \frac{e^{tx}}{t} \right|_{4}^{6} < + \infty \quad \forall t \in \mathbb{R} \Rightarrow \exists M_X(t)
\]
\end{enumerate}

\end{ej}

\begin{ej}
Con objeto de establecer un plan de producción, una empresa ha estimado que la demanda de
sus clientes, en miles de unidades del producto, se comporta semanalmente con arreglo a una
ley de probabilidad dada por la función de densidad:

\[
f(x)=\frac{3}{4} (2x-x^2), \hspace{0.3cm} 0 \leq x \leq 2
\]

\begin{enumerate}
    \item[a)] ¿Qué cantidad deberá tener dispuesta a la venta al comienzo de cada semana para poder satisfacer plenamente la demanda con probabilidad 0.5?
    
    Se nos está pidiendo calcular el valor $x$ que toma la variable aleatoria continua $X$ ``cantidad dispuesta a la venta'' tal que $P(X\leq x) = F_{X}(x) = 0.5$, por lo que nos hace falta conocer la función de distribución de $X$. Al tratarse de una variable aleatoria continua, podemos calcularla a partir de la función de densidad de la siguiente forma: $F_{X}(x) = \int_{-\infty}^{x}f(t)dt$.
    
    \[
    F_{X}(x) = \left\{ \begin{array}{lcc}
             \int_{-\infty}^{x}0dt &   si  & x < 0 \\
             \\ \int_{-\infty}^{0}0dt + \int_{0}^{x}\frac{3}{4} (2t-t^2)dt &  si & 0 \leq x \leq 2 \\
             \\ \int_{-\infty}^{0}0dt + \int_{0}^{2}\frac{3}{4} (2t-t^2)dt + \int_{2}^{x}0dt &  si  & x > 2
             \end{array}
   \right.
    \]
    
    Desarrollamos las integrales:
    
    \[
    \int_{0}^{x}\frac{3}{4} (2t-t^2)dt = \left[\frac{3t^2}{4}-\frac{t^3}{4}\right]^{x}_{0} = \frac{3x^2}{4}-\frac{x^3}{4}
    \]
    
    \[
    \int_{0}^{2}\frac{3}{4} (2t-t^2)dt = \left[\frac{3t^2}{4}-\frac{t^3}{4}\right]^{2}_{0} = \frac{12}{4}-\frac{8}{4}=1
    \]
    
    Y la función de distribución queda así:
    
    \[
    F_{X}(x) = \left\{ \begin{array}{lcc}
             0 &   si  & x < 0 \\
             \\ \frac{3x^2}{4}-\frac{x^3}{4} &  si & 0 \leq x \leq 2 \\
             \\ 1 &  si  & x > 2
             \end{array}
   \right.
    \]
    
    Como $F_{X}$ solo puede tomar el valor 0.5 en el intervalo $[0,2]$, tendremos que resolver la siguiente ecuación:
    
    \[
    \frac{3x^2}{4}-\frac{x^3}{4} = \frac{1}{2} \rightarrow x^3-3x^2+2=0
    \]
    
    Y dicha ecuación tiene como soluciones: $x_1 = 1, x_2 = 1-\sqrt{3}, x_3 = 1+ \sqrt{3}$. De entre esas soluciones, podemos descartar $x_2$ ya que es menor que 0 y nuestra variable aleatoria solo toma valores positivos. También podemos descartar $x_3$ ya que es mayor que 2 y por lo tanto $P(X\leq x_3) = 1 \neq 0.5$, por lo que el $x$ que buscamos es $x = 1$, y podemos concluir que será necesario tener 1000 unidades disponibles a la venta desde el principio de la semana si se quiere satisfacer plenamente la demanda con probabilidad 0.5.

    \item[b)] Pasado cierto tiempo, se observa que la demanda ha crecido, estimándose que en ese momento se distribuye según la función de densidad:
    
    \[
    f(y)=\frac{3}{4} (4y-y^2-3), \hspace{0.3cm} 1 \leq y \leq 3
    \]
    
    Los empresarios sospechan que este crecimiento no ha afectado a la dispersión de la demanda, ¿es cierta esta sospecha?
    
    Para determinar si es cierta la sospecha, calcularemos los coeficientes de variación de las variables aleatorias $X$ e $Y$, para lo cual necesitaremos hallar $E[X], E[Y], Var(X)$ y $Var(Y)$:
    
    \begin{equation*}
        E[X] = \int_{-\infty}^{+\infty}x \cdot f(x)dx = \int_{-\infty}^{0}0dx + \int_{0}^{2}\frac{3}{4} (2x^2-x^3)dx + \int_{2}^{+\infty}0dx = \left[\frac{x^3}{2}-\frac{3x^4}{16}\right]^{2}_{0} = 1
    \end{equation*}
    
    \begin{equation*}
        E[Y] = \int_{-\infty}^{+\infty}y \cdot f(y)dx = \int_{-\infty}^{1}0dy + \int_{1}^{3}\frac{3}{4} (4y^2-y^3-3y)dy + \int_{3}^{+\infty}0dy = \left[y^3-\frac{3y^4}{16}-\frac{9y^2}{8}\right]^{3}_{1} = 2
    \end{equation*}
    
    \begin{equation*}
        E[X^2] = \int_{-\infty}^{+\infty}x^2 \cdot f(x)dx = \int_{-\infty}^{0}0dx + \int_{0}^{2}\frac{3}{4} (2x^3-x^4)dx + \int_{2}^{+\infty}0dx = \left[\frac{3x^4}{8}-\frac{3x^5}{20}\right]^{2}_{0} = 1.2
    \end{equation*}
    
    \begin{equation*}
        E[Y^2] = \int_{-\infty}^{+\infty}y^2 \cdot f(y)dx = \int_{-\infty}^{1}0dy + \int_{1}^{3}\frac{3}{4} (4y^3-y^4-3y^2)dy + \int_{3}^{+\infty}0dy =
    \end{equation*}
    
    \begin{equation*}
        = \left[\frac{3y^4}{4}-\frac{3y^5}{20}-\frac{3y^3}{4}\right]^{3}_{1} = 4.2
    \end{equation*}
    
    \begin{equation*}
        C.V.(X)=\frac{\sqrt{Var(X)}}{E[X]} = \frac{\sqrt{E[X^2]-E[X]^2}}{E[X]} = \frac{\sqrt{0.2}}{1} = \frac{\sqrt{5}}{5}
    \end{equation*}
    
    \begin{equation*}
        C.V.(Y)=\frac{\sqrt{Var(Y)}}{E[Y]} = \frac{\sqrt{E[Y^2]-E[Y]^2}}{E[Y]} = \frac{\sqrt{0.2}}{2} = \frac{\sqrt{5}}{10}
    \end{equation*}

A la vista de los coeficientes de variación obtenidos, las sospechas eran ciertas ya que estos no coinciden.
\end{enumerate}
\end{ej}
\end{document}